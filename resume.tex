%-------------------------
% Resume in LateX
% Author : Sourabh Bajaj
% License : MIT
%------------------------

\documentclass[letterpaper,11pt]{article}

\usepackage{latexsym}
\usepackage[empty]{fullpage}
\usepackage{titlesec}
\usepackage{marvosym}
\usepackage[usenames,dvipsnames]{color}
\usepackage{verbatim}
\usepackage{enumitem}
\usepackage{fancyhdr}
\usepackage[english]{babel}
\usepackage{tabularx}
\input{glyphtounicode}

\usepackage{settings}
\usepackage{xspace}
% \usepackage[style=numeric,maxnames=2]{biblatex}
\addbibresource{own-bib.bib}

\DeclareBibliographyAlias{article}{std}
\DeclareBibliographyAlias{book}{std}
\DeclareBibliographyAlias{booklet}{std}
\DeclareBibliographyAlias{collection}{std}
\DeclareBibliographyAlias{inbook}{std}
\DeclareBibliographyAlias{incollection}{std}
\DeclareBibliographyAlias{inproceedings}{std}
\DeclareBibliographyAlias{manual}{std}
\DeclareBibliographyAlias{misc}{std}
\DeclareBibliographyAlias{online}{std}
\DeclareBibliographyAlias{patent}{std}
\DeclareBibliographyAlias{periodical}{std}
\DeclareBibliographyAlias{proceedings}{std}
\DeclareBibliographyAlias{report}{std}
\DeclareBibliographyAlias{thesis}{std}
\DeclareBibliographyAlias{unpublished}{std}
\DeclareBibliographyAlias{*}{std}

\DeclareBibliographyDriver{std}{%
  \usebibmacro{bibindex}%
  \usebibmacro{begentry}%
  \usebibmacro{author/editor+others/translator+others}%
  \setunit{\labelnamepunct}\newblock
  \usebibmacro{title}%
  \newunit\newblock
  \usebibmacro{booktitle}%
  \newunit\newblock
  \usebibmacro{finentry}}

\myname{Ananthanarayanan}{Sundaram}

\pagestyle{fancy}
\fancyhf{} % Clear all header and footer fields
\fancyfoot{}
\renewcommand{\headrulewidth}{0pt}
\renewcommand{\footrulewidth}{0pt}

% Adjust margins
\addtolength{\oddsidemargin}{-0.5in}
\addtolength{\evensidemargin}{-0.5in}
\addtolength{\textwidth}{1in}
\addtolength{\topmargin}{-.5in}
\addtolength{\textheight}{1.0in}

\urlstyle{same}

\raggedbottom
\raggedright
\setlength{\tabcolsep}{0in}

% Sections formatting
\titleformat{\section}{
  \vspace{-4pt}\scshape\raggedright\large
}{}{0em}{}[\color{black}\titlerule \vspace{-5pt}]

% Ensure that generate PDF is machine readable/ATS parsable
\pdfgentounicode=1

%-------------------------
% Custom commands
\newcommand{\resumeItem}[2]{
  \item\small{
    \textbf{#1}{: #2 \vspace{-2pt}}
  }
}

% Just in case someone needs a heading that does not need to be in a list
\newcommand{\resumeHeading}[4]{
    \begin{tabular*}{0.99\textwidth}[t]{l@{\extracolsep{\fill}}r}
      \textbf{#1} & #2 \\
      \textit{\small#3} & \textit{\small #4} \\
    \end{tabular*}\vspace{-5pt}
}

\newcommand{\resumeSubheading}[4]{
  \vspace{-1pt}\item
    \begin{tabular*}{0.97\textwidth}[t]{l@{\extracolsep{\fill}}r}
      \textbf{#1} & #2 \\
      \textit{\small#3} & \textit{\small #4} \\
    \end{tabular*}\vspace{-5pt}
}

\newcommand{\resumeSubSubheading}[2]{
    \begin{tabular*}{0.97\textwidth}{l@{\extracolsep{\fill}}r}
      \textit{\small#1} & \textit{\small #2} \\
    \end{tabular*}\vspace{-5pt}
}

\newcommand{\resumeSubItem}[2]{\resumeItem{#1}{#2}\vspace{-4pt}}

\renewcommand{\labelitemii}{$\circ$}

\newcommand{\resumeSubHeadingListStart}{\begin{itemize}[leftmargin=*]}
\newcommand{\resumeSubHeadingListEnd}{\end{itemize}}
\newcommand{\resumeItemListStart}{\begin{itemize}}
\newcommand{\resumeItemListEnd}{\end{itemize}}
\newcommand{\resumeProjectTask}{\begin{itemize}[label=\textbf{--}]}
% \newcommand{\endProjectTask}{\end{itemize}}

\newcommand{\sq}{\textit{SubmitQueue}\xspace}
\newcommand{\uci}{\textit{uCI}\xspace}

\newcommand{\resumeTalkTitle}[1]{
  \item{
    \textbf{#1} 
  }\vspace{-5pt}
}

\newcommand{\resumeTalk}[2]{
  \item
    \textit{#1}
}

%-------------------------------------------
%%%%%%  CV STARTS HERE  %%%%%%%%%%%%%%%%%%%%%%%%%%%%


\begin{document}

%----------HEADING-----------------
\begin{tabular*}{\textwidth}{l@{\extracolsep{\fill}}r}
  \textbf{\Large{Sundaram Ananthanarayanan}} & \href{mailto:me@sundaram.io}{me@sundaram.io}\\
  \href{https://sundaram.io/about-me}{https://sundaram.io/about-me} & +1(650)-666-9264 \\
\end{tabular*}


%-----------EDUCATION-----------------
\section{Education}
\resumeSubHeadingListStart
\resumeSubheading
{Stanford University}{Stanford, CA}
{Master of Science in Electrical Engineering;  \textit{GPA: 3.9/4.0}}{Sep. 2012 -- Jun. 2014}
\resumeSubheading
{College of Engineering, Guindy, Anna University}{Chennai, India}
{Bachelor of Engineering in Information Technology;  GPA: 9.32/10.0}{Aug. 2008 -- June. 2012}
\resumeSubHeadingListEnd


%-----------EXPERIENCE-----------------
\section{Working Experience}
\resumeSubHeadingListStart

\resumeSubheading
{Netflix}{Los Gatos, CA}
{Staff Software Engineer, Data Platform}{Dec 2019 - Present}
\resumeItemListStart
\resumeItem{\href{https://netflix.github.io/mantis/}{Mantis}}
{
  Mantis is a stream processing engine developed at Netflix, designed to address the unique challenges posed by operational data.
  Over the past couple of years, I have been leading its development.
  \begin{itemize}[label=\textbf{--}]
    \item
          Transitioned Mantis from Apache Mesos to Kubernetes with an innovative architecture.
          Guided a team of senior engineers from prototype to production and open-sourcing.
          % \item
          %       Directed a year-long migration effort, moving thousands of Mantis Jobs to Kubernetes without user disruption.
    \item
          Reduced annual compute costs by millions through ML-based container optimizations.
    \item
          Championed Mantis adoption to other companies including Stripe.
  \end{itemize}
}
\resumeItem{Flink}
{
  I have also contributed to the Flink ecosystem at Netflix - a stream processing engine for analytical needs.
  \begin{itemize}[label=\textbf{--}]
    % Teams building stream processing pipelines have to maintain separate batch jobs that pull data from warehouse during outages.
    \item Created a \href{https://www.youtube.com/watch?v=tB4rx_W9Xqw}{system} enabling users to backfill Flink pipelines using Apache Iceberg, eliminating the need for separate batch jobs to pull data from the warehouse during outages.
    \item Designed the system to mimic Kafka properties when reading from Data Lakes, ensuring effortless integration.
    \item The system has been adopted by hundreds of pipelines within Netflix. Helped open-source the project, now utilized by other companies using Apache Iceberg.
  \end{itemize}
}
\resumeItemListEnd

% --------Multiple Positions Heading------------
%  \resumeSubSubheading
%   {Software Engineer I}{Oct 2014 -- Sep 2016}
%   \resumeItemListStart
%      \resumeItem{Apache Beam}
%        {Apache Beam is a unified model for defining both batch and streaming data-parallel processing pipelines}
%   \resumeItemListEnd

%-------------------------------------------

\resumeSubheading
{Uber}{San Francisco, CA}
{Senior Software Engineer II, Developer Platform}{May 2016 - Dec 2019}
\resumeItemListStart
\resumeItem{\href{https://vimeo.com/358691692}{SubmitQueue}}
{
  1000s of engineers committing changes concurrently to a repository leads to frequent master breakages.
  Explored \& conceived a new system called \sq that guarantees an \textbf{always-green} master at scale. At Uber, \sq handles 1000s of commits/hr submitted by 1000s of engineers every day.
  \begin{itemize}[label=\textbf{--}]
    \item
          Led a team of 5 engineers to build the system: reading papers on state-of-the-art techniques used in similar domains such as Databases, experimented with various approaches to find a scalable solution, \& architected the system to handle 1000s of changes/hr.
    \item
          Published a research paper presenting the design \& implementation of \sq at \href{https://doi.org/10.1145/3302424.3303970}{\textbf{Eurosys'19}}.
  \end{itemize}
}
\resumeItem{uCI}
{
  Because existing open-source CI systems such as \href{https://jenkins.io/}{Jenkins} did not scale to Uber's needs, I helped build \uci - a distributed system to handle reliable execution of millions of stateful tasks every day on 1000s of CI machines.
  \begin{itemize}[label=\textbf{--}]
    \item
          Led a team of 6 engineers to design a state-of-the-art cluster scheduler that handles faults gracefully (\textit{reliability}), exploits data locality to come up with optimal placements (\textit{performance}), scales horizontally on every layer (\textit{scalability}), and finally guarantees isolation at task/resource levels.
          % \item
          %       Designed the system leveraging existing open-source technologies such as \href{http://mesos.apache.org/}{Apache Mesos} for cluster management, \href{https://cadenceworkflow.io/}{Cadence} for workflow orchestration \& \href{https://www.docker.com/}{Docker} for executing tasks in a containerized environment.
          % \item
          %       Sped up build times for Android CI workflows by 4-6x, reducing hour-long workflows to order of minutes.
  \end{itemize}
}
\resumeItemListEnd

\resumeSubheading
{Baidu Research Silicon Valley AI Lab}{Sunnyvale, CA}
{Software Engineer}{Jan 2016 - May 2016}
\resumeItemListStart
\resumeItem{Speech Recognition}
{Designed \& productionized deep-learning based Speech Recognition APIs which power Android apps such as \href{https://techcrunch.com/2016/10/03/baidus-new-talktype-keyboard-app-emphasizes-voice-input-over-typing/}{TalkType}. }.
\resumeItemListEnd

\resumeSubheading
{Twitter Inc}{San Francisco, CA}
{Software Engineer}{Jun 2014 - Jan 2016}
\resumeItemListStart
\resumeItem{AddressBook Infrastructure}
{
  Engineered a system to store and retrieve contacts from the phone books of Twitter's 300M+ Monthly Active Users (MAUs).
}
\resumeItemListEnd

\resumeSubHeadingListEnd

%-----------PREVIOUS-----------------
\section{Ancient History}
\resumeSubHeadingListStart
\resumeSubheading
{Microsoft}{Redmond, WA}
{Software Engineering Intern, Kernel Core}{Jun 2013 - Sep 2013}
\resumeSubheading
{Google Summer of Code}{Chennai, India}
{Worked on Metalink Support for Google Chrome}{Jun 2012 - Sep 2012}
\resumeSubheading
{University of Waterloo}{Waterloo, Canada}
{Research Intern - Worked on design \& application of One-Instruction Processors}{Apr. 2011 -- June. 2011}
\resumeSubHeadingListEnd

% --------SELECTED PUBLICATIONS------------
\section{Selected Publications}
\nocite{*}
\printbibliography[heading={none}]


%-----------TALKS-----------------
\section{Selected Talks}
\resumeSubHeadingListStart
\resumeTalkTitle{\href{https://www.sundaram.io/slides/dais22.pdf}{Backfilling Streaming Data Pipelines using Kappa Architecture}}
{
  \resumeItemListStart
  \resumeTalk {Databricks Data + AI Summit, June 2022} {San Francisco, CA}
  \resumeTalk {LinkedIn, March 2022} {Mountain View, CA}
  \resumeTalk {Flink Forward, Nov 2021} {San Francisco, CA}
  \resumeItemListEnd
}
\resumeTalkTitle{\href{https://www.sundaram.io/slides/eurosys19.pdf}{Keeping Master Green at Scale}}
{
  \resumeItemListStart
  \resumeTalk {Twitter, Jan 2022} {San Francisco, CA}
  \resumeTalk {Google Journal Club, May 2019} {San Francisco, CA}
  \resumeTalk {Facebook, Jan 2019} {Menlo Park, CA}
  \resumeItemListEnd
}
\resumeSubHeadingListEnd


% %-----------PROJECTS-----------------
% \section{Projects}
% \resumeSubHeadingListStart
% \resumeSubItem{QuantSoftware Toolkit}
% {Open source Python library for financial data analysis and machine learning for finance.}
% \resumeSubItem{GitHub Visualization}
% {Data visualization of Git log data using D3 to analyze project trends over time.}
% \resumeSubItem{Recommendation System}
% {Music and movie recommender systems using collaborative filtering on public datasets.}
% \resumeSubItem{Mac Setup}
% {Book that gives step-by-step instructions on setting up developer environment on macOS.}
% \resumeSubHeadingListEnd

%
%--------SKILLS------------
\section{Skills}
\resumeSubHeadingListStart
\item
\textbf{Languages}{: Java, Python, Scala, C++}
\item \vspace{-5pt}
\textbf{Interests}{: Distributed Systems, Stream Processing, Machine Learning, Reinforcement Learning}
\resumeSubHeadingListEnd


%-------------------------------------------
\end{document}